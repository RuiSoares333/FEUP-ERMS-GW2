
HelixRM proves to be a valuable requirements management tool for software
and system engineering projects, offering several essential features, as
previously stated.

Its ability to integrate with common documentation formats improves
usability and facilitates collaboration among various participants,
such as stakeholders and domain experts.
This, in turn, streamlines the development process and reduces communication challenges\cite{michele_zamparelli_2006}.


The following features are some of the advantages of using HelixRM in software and system engineering projects:

\paragraph{Rapid Development}
Facilitates rapid development, helping to avoid costly delays and significantly improve time to market\footnotemark[1].

\paragraph{Scalability}
Highly scalable to support evolving space systems, easy and straightforward to extend or build new HELIX applications\footnotemark[1].

\paragraph{Flexibility}
Exceptionally flexible to support extensive configuration or integration with 3rd party software\footnotemark[1].

\paragraph{Compatibility}
Compatible with both Agile and traditional methodologies\footnotemark[2].

\paragraph{Traceability}
Provides detailed reports and metrics for project monitoring\footnotemark[2].

\paragraph{Integration}
Supports integration with other tools and systems, such as JIRA\footnotemark[2].

These features end up contributing to a better project quality,
efficiency, and adaptability in development environments\cite{youngki_hong__2010}.


\footnotetext[1]{\cite{bright_ascention_2024}}
\footnotetext[2]{\cite{helix_alm_review}}





